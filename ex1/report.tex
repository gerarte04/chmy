\documentclass[a4paper,12pt,titlepage,finall]{article}

\usepackage[T1,T2A]{fontenc}     % форматы шрифтов
\usepackage[utf8x]{inputenc}     % кодировка символов, используемая в данном файле
\usepackage[russian]{babel}      % пакет русификации
\usepackage{tikz}                % для создания иллюстраций
\usepackage{pgfplots}            % для вывода графиков функций
\usepackage{geometry}		 % для настройки размера полей
\usepackage{indentfirst}         % для отступа в первом абзаце секции
\usepackage{amsmath,amssymb}

% выбираем размер листа А4, все поля ставим по 3см
\geometry{a4paper,left=30mm,top=30mm,bottom=30mm,right=30mm}

\setcounter{secnumdepth}{0}      % отключаем нумерацию секций

\usepgfplotslibrary{fillbetween} % для изображения областей на графиках

\begin{document}
% Титульный лист
\begin{titlepage}
    \begin{center}
	{\small \sc Московский государственный университет \\имени М.~В.~Ломоносова\\
	Факультет вычислительной математики и кибернетики\\}
	\vfill
	{\Large \bf Отчет по заданию №1 по курсу}\\
	~\\
	{\Large \bf <<Численные методы линейной алгебры>>}\\ 
    \end{center}
    \begin{flushright}
	\vfill {Выполнил:\\
	студент 304 группы\\
	Гераськин~А.~Ю.}
    \end{flushright}
    \begin{center}
	\vfill
	{\small Москва\\2024}
    \end{center}
\end{titlepage}

% Автоматически генерируем оглавление на отдельной странице
\tableofcontents
\newpage

\section{Постановка задачи}

Требуется двумя различными методами получить QR-разложение данной матрицы $A \in \mathbb{R}^{n \times n}$ и сравнить полученные разложения, вычислив матричную норму разности $||A \, - \, QR||$ для каждого построенного разложения. Используется матричная норма, подчиненная максимум-норме арифметического пространства $\mathbb{R}^{n}$.

Далее необходимо с помощью генератора псевдослучайных чисел построить вектор $X \in \mathbb{R}^{n}$ с компонентами $x_k \in [-1,1], \, k = 1, \ldots, n$ и решить систему $A\overline{x} = f$ с правой частью $f = Ax$ при помощи полученных разложений, а также вычислить максимум-норму невязки $\displaystyle r = A\overline{x}\, - \,f$ и максимум-норму погрешности решения $\displaystyle \delta = \overline{x}\,-\,x$.

В данном варианте используются метод Холецкого (метод квадратного корня) и метод отражений Хаусхолдера.

\subsection{Метод Холецкого}

Данный метод заключается в представлении симметричной положительно определённой матрицы $\displaystyle L$ в виде $\displaystyle A=LL^{T}$, где $\displaystyle L$ — нижняя треугольная матрица со строго положительными элементами на диагонали.

Элементы матрицы $\displaystyle L$ можно вычислить, начиная с верхнего левого угла матрицы, по формулам

\[\displaystyle {\begin{aligned}l_{11}&={\sqrt {a_{11}}},\\l_{j1}&={\frac {a_{j1}}{l_{11}}},\quad j\in [2,n],\\l_{ii}&={\sqrt {a_{ii}-\sum _{p=1}^{i-1}l_{ip}^{2}}},\quad i\in [2,n],\\l_{ji}&={\frac {1}{l_{ii}}}\left(a_{ji}-\sum _{p=1}^{i-1}l_{ip}l_{jp}\right),\quad i\in [2,n-1],j\in [i+1,n].\end{aligned}}\]

Выражение под корнем всегда положительно, если $\displaystyle A$ — действительная положительно определённая матрица. Вычисление происходит сверху вниз, слева направо, т. е. сперва $\displaystyle L_{ij}$, а затем $\displaystyle L_{ii}$.

Выполнив разложение $\displaystyle A=LL^{T}$, решение $\displaystyle x$ можно получить последовательным решением двух треугольных систем уравнений: $\displaystyle Ly=b$ и $\displaystyle L^{T}x=y$ (это нетрудно сделать, т. к. матрицы $L$ и $L^T$ треугольные). Такой способ решения иногда называется методом квадратных корней. По сравнению с более общими методами, такими как метод Гаусса или LU-разложение, он устойчивее численно и требует вдвое меньше арифметических операций.

\subsection{Метод Хаусхолдера}

Метод Хаусхолдера (метод отражений) используется для разложения матриц в виде $A=QR$ ($Q$ — унитарная, $R$ — верхняя треугольная матрица). При этом матрица $Q$ хранится и используется не в своём явном виде, а в виде произведения матриц отражения. Каждая из матриц отражения может быть определена одним вектором. Это позволяет в классическом исполнении метода отражений хранить результаты разложения на месте матрицы $A$ с использованием минимального одномерного дополнительного массива.

Для выполнения QR-разложения матрицы используются умножения слева её текущих модификаций на матрицы Хаусхолдера (отражений) — матрицы вида $P=I-2uu^T$, где $u$ — вектор, удовлетворяющий равенству $u^{T}u=1$. Является одновременно унитарной ($P^{T}P=I$) и эрмитовой ($P^{T}=P$), поэтому обратна самой себе ($P^{-1}=P$).

На $i$-м шаге метода с помощью преобразования отражения убираются ненулевые поддиагональные элементы в $i$-м столбце. Таким образом, после $n-1$ шагов преобразований получается матрица $R$ из QR-разложения.\\

{\large \bf Алгоритм}\\

Инициализация:
\[\begin{aligned}&Q = I\\
&R = A\end{aligned}\]

На каждой итерации для $k = 1, \ldots, n - 1$:
\[\displaystyle {\begin{aligned}&x = a_k \\&x_1, \ldots ,x_{k-1} = 0 \\&u = x - ||x|| e\\&P = I - \frac{2uu^{T}}{||u||^2}\\&Q = QP\\&A = PA\\&R = A\end{aligned}}\]

где $e$ - орт, у которого на каждой итерации только k-ый элемент равен 1.\\

После получения разложения решение $x$ можно получить, решив систему с верхнетреугольной матрицей \[QRx=f \iff Rx=Q^{T}f\] (поскольку $Q^{-1}=Q^T$).

\subsection{Алгоритм решения системы уравнений с треугольной матрицей}

Для СЛАУ $Ax=f$ с верхнетреугольной матрицей $A$: для $i = n,\,\ldots\,,1$
\[x_i=\frac{f_i-\displaystyle\sum_{j=i+1}^{n} a_{ij}x_j}{a_{ii}}.\]

Для СЛАУ $Ax=f$ с нижнетреугольной матрицей $A$: для $i = 1,\,\ldots\,,n$
\[x_i=\frac{f_i-\displaystyle\sum_{j=1}^{i - 1} a_{ij}x_j}{a_{ii}}.\]

\section{Сравнение методов}

Оба метода разложения, а также решение системы уравнений при помощи разложений были реализованы программно с замерением времени работы. Результаты вычислений приведены в таблицах ниже.

\begin{table}[h]
\centering
\begin{tabular}{|c|c|c|}
\hline
Метод & $||A\,-\,QR||$ & Время, мс \\
\hline
Холецкий & 1.45439e-14 & 0.447019 \\
\hline
Хаусхолдер & 3.71744e-13 & 149.285 \\
\hline
\end{tabular}
\caption{Результаты разложений}
\label{table1}
\end{table}

\begin{table}[h]
\centering
\begin{tabular}{|c|c|c|c|}
\hline
Тест & $||A\overline{x}\,-\,Ax||$ & $||\overline{x}\,-\,x||$ & Время, мс \\
\hline
1 & 9.9476e-14 & 1.44329e-15 & 0.028233 \\
\hline
2 & 8.52651e-14 & 1.44329e-15 & 0.027121 \\
\hline
3 & 9.9476e-14 & 1.33227e-15 & 0.027471 \\
\hline
4 & 7.10543e-14 & 8.88178e-16 & 0.027291 \\
\hline
5 & 8.52651e-14 & 1.22125e-15 & 0.027071 \\
\hline
Среднее & 8.81073e-14 & 1.26565e-15 & 0.027437 \\
\hline
\end{tabular}
\caption{Результаты решения СЛАУ при помощи метода Холецкого}
\label{table1}
\end{table}

\begin{table}[h]
\centering
\begin{tabular}{|c|c|c|c|}
\hline
Тест & $||A\overline{x}\,-\,Ax||$ & $||\overline{x}\,-\,x||$ & Время, мс \\
\hline
1 & 5.96856e-13 & 8.54872e-15 & 0.025909 \\
\hline
2 & 3.48166e-13 & 5.55112e-15 & 0.024757 \\
\hline
3 & 6.89226e-13 & 9.65894e-15 & 0.024596 \\
\hline
4 & 3.90799e-13 & 6.55032e-15 & 0.024356 \\
\hline
5 & 3.97904e-13 & 5.32907e-15 & 0.024296 \\
\hline
Среднее & 4.8459e-13 & 7.12763e-15 & 0.024782 \\
\hline
\end{tabular}
\caption{Результаты решения СЛАУ при помощи метода Хаусхолдера}
\label{table1}
\end{table}

Как можно видеть, метод Хаусхолдера работает гораздо медленнее, чем метод Холецкого. Это связано с тем, что он вычислительно сложнее, поскольку в нем выполняется большое количество перемножений и вычитаний матриц, в то время как метод Холецкого сразу заполняет значения матриц без промежуточных вычислений. Этот результат можно улучшить, если оптимизировать аллокации памяти при операциях над матрицами, но не существенно.

В плане же точности оба метода показали себя хорошо — абсолютные погрешности всех полученных значений не превышают $10^{-13}$. Поэтому каждый из этих методов имеет место быть и может успешно применяться в различных задачах.

\section{Сборка и запуск программы}

Для сборки программы необходима установленная утилита CMake версии не ниже 3.15, а также поддержка стандарта C++17.

Команда сборки:
\begin{verbatim}
cmake . && cmake --build .
\end{verbatim}

У собранной программы поддерживаются следующие аргументы командной строки:

\begin{itemize}
\item \texttt{-i}, \texttt{-{}-input}: имя файла со входной матрицей (обязательный аргумент);
\item \texttt{-o}, \texttt{-{}-output}: имя файла, в который будет записан вывод программы (по умолчанию вывод производится на стандартный выходной поток);
\item \texttt{-{}-qr-method}: метод разложения матрицы (обязательный аргумент, значения cholesky/householder);
\item \texttt{-n}, \texttt{-{}-n-tests}: количество тестов для решения СЛАУ (натуральное число, по умолчанию 5).
\end{itemize}

Пример запуска программы:
\begin{verbatim}
./QrDecompositionSolver -i SLAU_var_2.csv --qr-method cholesky
\end{verbatim}

\end{document}
